\section{Basics}
\makesubcontentsslides

\begin{frame}
  \begin{block}{Basic Type}\pause
  \begin{itemize}
    \item Integer
    \item Real/double
    \item Complex/double complex
    \item Logical
    \item Character
  \end{itemize}
  \end{block}
\end{frame}


\subsection{Numeric Types}

\begin{frame}
  \begin{block}{Integer}\pause
  Whole numbers.
  \begin{itemize}
    \item 1
    \item -517
    \item 0
    \item Not 1.2
    \item Not $\pi$
    \item Not 0.0
  \end{itemize}
  \end{block}
\end{frame}



\begin{frame}
  \begin{block}{Real and Double}\pause
  Floating point
  \begin{itemize}
    \item 1.1
    \item 3.14159
    \item Not 1
    \item Not $\sqrt{-1}$
  \end{itemize}
  \end{block}
\end{frame}


\begin{frame}
  \begin{block}{Basic Numeric Operations}\pause
  \begin{itemize}
    \item \tabinlist{\code{a=b}:} {assign the value of \code{b} to \code{a}}
    \item \tabinlist{\code{a+b}:}{add \code{a} and \code{b}}
    \item \tabinlist{\code{a-b}:} {subtract \code{b} from \code{a}}
    \item \tabinlist{\code{a*b}:} {multiply \code{a} and \code{b}}
    \item \tabinlist{\code{a/b}:} {divide \code{b} into \code{a}}
    \item \tabinlist{\code{a**b}:} {raise \code{a} to the power \code{b}}
  \end{itemize}
  \end{block}
\end{frame}


\begin{frame}[fragile]
  \begin{block}{Quick example}\pause
\begin{lstlisting}
program arithmetic
    integer :: a = 2, b = 3

    print *, a+b
    print *, a**b
    print *, a/b
end program
\end{lstlisting}
\begin{lstlisting}[language=shl]
           5
           8
           0
\end{lstlisting}
  \end{block}
\end{frame}



\subsection{Logical Type}

\begin{frame}
  \begin{block}{Logical}\pause
  Logical variables can take two values:
  \begin{itemize}
    \item \code{.true.}
    \item \code{.false.}
  \end{itemize}
  \end{block}
\end{frame}

\begin{frame}
  \begin{block}{Comparing Logicals}\pause
  \begin{itemize}
    \item \tabinlist{\code{.eqv.}} {tests if two logical expressions are equivalent}
    \item \tabinlist{\code{.neqv.}} {tests if two logical expressions are \emph{not} equivalent}
    \item \tabinlist{\code{.and.}} {the and operator}
    \item \tabinlist{\code{.or.}} {the or operator}
    \item \tabinlist{\code{.not.}} {the negation operator}
  \end{itemize}
  \end{block}
\end{frame}

\begin{frame}
  \begin{block}{Comparing Numerics}\pause
  \begin{itemize}
    \item \tabinlisttwo{\code{a<b} or \code{a.lt.b}:} {\code{a} less than \code{b}}
    \item \tabinlisttwo{\code{a<=b} or \code{a.le.b}:} {\code{a} less than or equal to \code{b}}
    \item \tabinlisttwo{\code{a>b} or \code{a.gt.b}:} {\code{a} greater than \code{b}}
    \item \tabinlisttwo{\code{a>=b} or \code{a.ge.b}:} {\code{a} greater than or equal to \code{b}}
    \item \tabinlisttwo{\code{a==b} or \code{a.eq.b}:} {\code{a} equal to \code{b}}
    \item \tabinlisttwo{\code{a/=b} or \code{a.ne.b}:} {\code{a} not equal to \code{b}}
  \end{itemize}
  The type of a numeric comparison is of type \code{logical}.
  \end{block}
\end{frame}

\begin{frame}
  \begin{block}{Comparing Numerics}\pause
  Note: The output of a comparison of numerics is logical:
  \begin{itemize}
    \item \code{a < b < c} makes no sense (types mismatch)
    \item Instead: \code{a < b .and. b < c}
  \end{itemize}
  \end{block}
\end{frame}

\begin{frame}[fragile]
  \begin{block}{Quick example}\pause
\begin{lstlisting}
program logicals
    integer :: a = 2, b = 3, c = 1

    print *, a < b
    print *, a /= b .and. a < c
end program
\end{lstlisting}
\begin{lstlisting}[language=shl]
 T
 F
\end{lstlisting}
  \end{block}
\end{frame}

\begin{frame}
  \begin{block}{Implicit Declaration}\pause
  \begin{itemize}
    \item In Fortran, variables may be used \emph{implicitly}.
    \item Do not get into the habit of doing this.
    \item You can turn this off in a program (function, subroutine) by declaring \code{implicit none}.
    \item Declaring \code{implicit none} is generally recommended.
  \end{itemize}
  \end{block}
\end{frame}

\begin{frame}[fragile]
  \begin{block}{Implicit variables quick example}\pause
Compiles:
\vspace*{-.4cm}
\begin{lstlisting}
program implicit_declaration
    a = 2
    b = 3
    print *, a+b
end program
\end{lstlisting}
Fails to compile:
\vspace*{-.4cm}
\begin{lstlisting}
program implicit_declaration
    implicit none
    a = 2
    b = 3
    print *, a+b
end program
\end{lstlisting}
  \end{block}
\end{frame}


\begin{frame}
  \begin{block}{Example}\pause
    Go to example.\\[.4cm]
    Not covered: 
    \begin{itemize}
      \item complex
      \item character
      \item kind/precision
    \end{itemize}
  \end{block}
\end{frame}

% \begin{frame}[fragile]
%   \begin{block}{Logical}\pause
% \begin{lstlisting}[language=ft]
% 
% \end{lstlisting}
%   \end{block}
% \end{frame}

