\section{Introduction}
\makesubcontentsslides


\subsection{Background}

\begin{frame}
  \begin{block}{What is Fortran?}\pause
  \begin{itemize}
    \item \textbf{For}mula \textbf{tran}slation system.
    \item General purpose programming language.
    \item Well-suited for mathematics and engineering.
    \item Compiled (rather than interpreted)
    \item Portable.
  \end{itemize}
  \end{block}
\end{frame}

\begin{frame}
  \begin{block}{History}\pause
  \begin{itemize}
    \item Created at IBM in 1954 as an alternative to assembly language.
    \item Debuted in the days of punch cards.
    \item Regularly gets updated to a new standard.
  \end{itemize}
  \end{block}
\end{frame}

\begin{frame}[fragile]
\begin{block}{What people think Fortran is like}
\begin{minipage}{.475\textwidth}
\begin{lstlisting}
 1121 FORMAT(I4,F8.3)
 3298 CONTINUE
      IF(MOD(I,A)
     $.EQ.Z) THEN
      GOTO 2359
      ELSE IF(MOD(I,B)
     $.EQ.Z) THEN
      GOTO 8125
      ELSE
      WRITE (*,2930) I
      GOTO 7365
      END IF
 7235 FORMAT(A,F5.3)
 7356 CONTINUE
      I=I
     $+1
      GOTO 1249
 2930 FORMAT(I4,$)
 2359 CONTINUE
\end{lstlisting}
\end{minipage}
\begin{minipage}{.475\textwidth}
  \begin{center}
    \includegraphics{pics/tombstone}
  \end{center}
\end{minipage}
\end{block}
\end{frame}


\begin{frame}
  \begin{block}{Standards}\pause
  Fortran standards are denoted by year.
  \begin{itemize}
    \item Fortran 66
    \item Fortran 77
    \item Fortran 90
    \item Fortran 95
    \item Fortran 2003
    \item Fortran 2008
    \item Fortran 2015
  \end{itemize}
  \end{block}
\end{frame}


\begin{frame}
  \begin{block}{Language Features}\pause
  Fortran essentially has two formats:  Fortran 77 and ``modern''.\\
  All versions have:
  \begin{itemize}
    \item Matrices.
    \item Complex numbers.
    \item Good interoperability with C.
    \item REALLY good compilers.
  \end{itemize}
  Modern variants of Fortran additionally have:
  \begin{itemize}
    \item Dynamic memory allocation.
    \item Pointers.
    \item OOP.
  \end{itemize}
  \end{block}
\end{frame}

\begin{frame}
  \begin{block}{Applications for Fortran}\pause
  \begin{minipage}{.475\textwidth}
  Bad applications:
    \begin{itemize}
      \item A website
      \item An OS kernel
      \item Text processing
      \item Random number generators
    \end{itemize}
  \end{minipage}
  \begin{minipage}{.475\textwidth}
  Good applications:
    \begin{itemize}
      \item Mathematics
      \item Engineering
      \item Statistics
      \item Scientific computing
    \end{itemize}
  \end{minipage}
  \end{block}
\end{frame}

\begin{frame}
  \begin{block}{Compilers}\pause
  \begin{itemize}
    \item GNU \code{gfortran} (free)
    \item Intel \code{ifortran}
    \item Portland Group \code{pgfortran}
    \item \dots
  \end{itemize}
  \end{block}
\end{frame}


\begin{frame}
  \begin{block}{Why Fortran?}\pause
  \begin{itemize}
    \item High level language
    \item Avoids fiddly memory management like in C
    \item Fortran binaries are VERY fast
    \item Good HPC support (MPI, OpenMP, \dots)
  \end{itemize}
  \end{block}
\end{frame}



\subsection{Hello World}

\begin{frame}[fragile]
\begin{block}{Hello world example}
\begin{lstlisting}[title=hello.f]
      program helloworld
      print *, "Hello world"
      end program helloworld
\end{lstlisting}
\end{block}
\end{frame}

\begin{frame}[fragile]
\begin{block}{Hello world example}
\begin{lstlisting}[title=hello.f90]
program helloworld
    print *, "Hello world"
end program helloworld
\end{lstlisting}
\end{block}
\end{frame}

\begin{frame}
  \begin{block}{Comments}
    \begin{itemize}
      \item \code{.f} is for F77, \code{.f90} is for F90+
      \item Don't use F77.
      \item Fortran is case insensitive.
      \item \code{print} and \code{write} print/write output
      \item \code{read} reads input
    \end{itemize}
  \end{block}
\end{frame}


