\section[Functions]{Functions, Intrinsics, and Subroutines}
\makesubcontentsslides



\subsection{Functions}

\begin{frame}[fragile]
  \begin{block}{Functions}\pause
\begin{lstlisting}
! Declaration
function foo(bar)
    type :: foo
    ! statements
end function

! Invocation
a = foo(b)
\end{lstlisting}
  \begin{itemize}
    \item Can take variety of inputs.
    \item Returns single output.
  \end{itemize}
  \end{block}
\end{frame}

\begin{frame}[fragile]
  \begin{block}{Functions quick example}\pause
\begin{lstlisting}
function circumference(r)
    implicit none
    real :: pi = 3.14159
    real :: r
    real :: circumference
    circumference = 2.0*pi*r
end function circumference

program circles
    real :: r = 2.0
    real :: circumference
    print *, circumference(r)
end program circles
\end{lstlisting}
\begin{lstlisting}[language=shl]
   12.5663605  
\end{lstlisting}
  \end{block}
\end{frame}


\subsection{Intrinsics}

\begin{frame}
  \begin{block}{Intrinsics}\pause
  \begin{itemize}
    \item Built-in functions.
    \item Casting.
    \item Basic math utilitis.
    \item Bit-shifting.
  \end{itemize}
  \end{block}
\end{frame}

\begin{frame}
  \begin{block}{Intrinsic Examples}\pause
  {\small
    \begin{tabular}{ll|ll}\hline
      Intrinsic & Effect & Intrinsic & Effect\\\hline
      \code{int} & Convert to \code{integer} & \code{mod} & Modular arithmetic\\
      \code{real} & Convert to \code{real} & \code{abs} & Absolute value\\
      \code{floor} & Greatest integer below & \code{sqrt} & Square root\\
      \code{ceiling} & Smallest integer above & \code{exp} & Exponential\\\hline
    \end{tabular}
  }
  \end{block}
\end{frame}

\begin{frame}[fragile]
  \begin{block}{Intrinsics quick example}\pause
\begin{lstlisting}
integer :: a = 2, b = 3

print *, mod(3, 2)
print *, real(a/b)
print *, real(a) / real(b)
 \end{lstlisting}
\begin{lstlisting}[language=shl]
           1
   0.00000000    
  0.666666687
\end{lstlisting}
  \end{block}
\end{frame}



\subsection{Subroutines}

\begin{frame}[fragile]
  \begin{block}{Subroutines}\pause
\begin{lstlisting}
! Declaration
subroutine foo(bar, baz)
    type :: bar, baz
    ! statements
end subroutine

! Invocation
call foo(a, b)
\end{lstlisting}
  \begin{itemize}
    \item Can take variety of inputs.
    \item Returns a variety of outputs (modifying values of inputs).
    \item Equivalent in C is \code{void} function.
  \end{itemize}
  \end{block}
\end{frame}

\begin{frame}[fragile]
  \begin{block}{Intent}\pause
  \begin{itemize}
    \item Can declare intention of use for variable.
    \item \code{intent(in)}, \code{intent(out)}, \code{intent(inout)}.
    \item Like \code{implicit none}, not strictly necessary, but useful.
  \end{itemize}
\begin{lstlisting}
subroutine foo(a, b)
    integer, intent(in) :: a
    integer, intent(out) :: b
    ! statements
end subroutine
\end{lstlisting}
  \end{block}
\end{frame}


\begin{frame}[fragile]
  \begin{block}{Subroutine quick example}\pause
\vspace*{-.4cm}
\begin{lstlisting}
subroutine circ_stuff(r, circumference, area)
    implicit none
    real :: pi = 3.14159
    real, intent(in) :: r
    real, intent(out) :: circumference, area
    circumference = 2.0 * pi * r
    area = pi * r * r 
end subroutine circ_stuff

program circles
    real :: r = 2.0, circumference, area
    call circ_stuff(r, circumference, area)
    print *, "Circumference = ", circumference
    print *, "Area = ", area
end program circles
\end{lstlisting}
\vspace{-.2cm}
\begin{lstlisting}[language=shl]
 Circumference =    12.5663605    
 Area =    12.5663605   
\end{lstlisting}
  \end{block}
\end{frame}


\begin{frame}
  \begin{block}{Example}\pause
    Go to example.
  \end{block}
\end{frame}
