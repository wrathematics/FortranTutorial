\section{Precision}
\makesubcontentsslides


\begin{frame}
  \begin{block}{Precision}\pause
    In C, there are two floating point types
    \begin{itemize}
      \item \code{float}
      \item \code{double}
    \end{itemize}
    In elder Fortran, the corresponding types are
    \begin{itemize}
      \item \code{real}
      \item \code{double precision}
    \end{itemize}
  \end{block}
\end{frame}


\begin{frame}
  \begin{block}{Scientific Notation and Precision}\pause
\begin{align*}
\text{number} = \text{mantissa}\times 10^{\text{exponent}}\\
1.234 = 1234\times 10^{-4}
\end{align*}
    \begin{itemize}
      \item \code{real}: 23 bit of mantissa, 8 bits of exponent, and 1 sign bit.
      \item \code{double precision}: 52 bit of mantissa, 11 bits of exponent, and 1 sign bit.
    \end{itemize}
    \begin{itemize}
      \item \code{real}: $\approx$ 6 decimal digits of precision
      \item \code{double precision}: $\approx$ 15 decimal digits of precision
    \end{itemize}
  \end{block}
\end{frame}


\subsection{Kind}

\begin{frame}[fragile]
  \begin{block}{Kind}\pause
    \begin{itemize}
      \item \code{kind} is a parameter that specifies the storage/precision of a type (beyond its default)
      \item \code{real(kind=4)}
      \item Different compilers assign different meaning to \code{kind}.
      \item Avoid this complexity with the intrinsic functions \code{selected_<type>_kind}
    \end{itemize}
    \begin{itemize}
      \item \code{real(kind=selected_real_kind(6))}
      \item \code{real(kind=selected_real_kind(15))} 
      \item \dots
    \end{itemize}

  \end{block}
\end{frame}

\begin{frame}[fragile]
  \begin{block}{Quick example}\pause
\begin{lstlisting}
program kind_example
    implicit none
    integer, parameter :: r15 = selected_real_kind(15)
    real :: x
    real(kind=r15) :: y
    
    x = 1.0
    y = 1.0
    
    print *, x
    print *, y
    
end program
\end{lstlisting}
\begin{lstlisting}[language=shl]
   1.00000000    
   1.0000000000000000  
\end{lstlisting}
  \end{block}
\end{frame}
